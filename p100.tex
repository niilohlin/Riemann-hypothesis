\documentclass{article}%[fleqn] kan vara snyggt
\usepackage{amsthm}
\usepackage{amsmath}
\usepackage{amsfonts}
\usepackage{graphicx}
\usepackage[swedish]{babel}
\usepackage[utf8]{inputenc}

\def \width {3mm}

\begin{document}
\title{p100}
\author{Linus Bergkvist}
\date{}
\maketitle
%\setlength\parindent{0pt}
%\paragraph{Introduktion:} 
\noindent
{\bf Introduktion}\\
	Riemannhypotesen går ut på att alla icke-triviala nollställen till Riemanns
	zeta-funktionligger på linjen $\frac 1 2 + i \cdot t$. De triviala
	nollställena är $-2, -4, -6, -8, -10 \ldots$

\paragraph{Sats 1:}
Det finns inga nollställen till Riemanns zeta-funktion med $Re > 1$. \\
\\
För $s$ med $Re(s) > 1$ definieras Riemanns zeta-funktion som 
$\zeta(s) = \sum_{n = 1}^\infty \frac {1} {n^s}$ %ska man ta och sätta \limits_? för snygghetens skull?
\begin{equation}
	\begin{aligned}
		&\zeta(s) = \frac 1 {1^s} + \frac 1 {2^s} + \frac 1 {3^s} + \frac 1 {4^s} ... \notag\\
		&\zeta(s) \cdot \frac 1 {2^s} = \frac 1 {2^s} + \frac 1 {4^s} + \frac 1 {6^s} + \frac 1 {8^s} ... \\
		&\zeta(s) - \frac {\zeta(s)} {2^s} = \frac 1 {1^s} + \frac 1 {3^s} + \frac 1 {5^s} ... &
		\hspace{\width} &\text{Vi har nu tagit bort alla multiplar av } \frac 1 {2^s} \\
		&\zeta(s) (1 - \frac 1 {2^s}) \cdot \frac 1 {3^s} = \frac 1 {3^s} + \frac 1 {9^s} + \frac 1 {15^s} ...\\
		&\zeta(s)(1 - \frac 1 {2^s})(1 - \frac 1 {3^s}) = \frac 1 {1^s} + \frac 1 {5^s} + \frac 1 {7^s} ... &
		\hspace{\width} &\text{Vi har nu tagit bort alla} \\
		& & \hspace{\width} &\text{återstående multiplar av } \frac 1 {3^s} \\
	\end{aligned}
	\phantom{\hspace{4cm}}
\end{equation}
Om man gör så med alla primtal får man: 
\begin{equation}
	\begin{aligned}
		\notag
		&\zeta(s) \cdot (1 - \frac 1 {2^s})(1 - \frac 1 {3^s})(1 - \frac 1 {5^s}) ... = \frac 1 {1^s} = 1 \Leftrightarrow \\  
		&\zeta(s)  = \frac 1 {(1 - \frac 1 {2^s})(1 - \frac 1 {3^s})...} = \prod_p \frac 1 {1 - p^{-s}}.\hspace{\width} Re(s) > 1 \\
		&\frac {\zeta(s)} {\prod_p \frac 1 {(1 - p^{-s})}} = 1 = \zeta(s) \cdot \prod_p (1 - p^{-s}) \\
	\end{aligned}
\end{equation}
\\
Vilket visar att $\zeta(s) \ne 0 \text{ För } Re(s) > 1$

\pagebreak
\paragraph{Definition: Fourierserie och Fouriertransform.\\}
Eulers formel ger att $e^{i \cdot x} = cos(x) + i \cdot sin(x)$ Men går det att uttrycka en funktion som en oändlig summa
med sinus och cosinusvågor?\\
Mer matematisk blir frågan om kan man skriva en funktion $f(t)$ som 
$$f(t) = \sum_{n = - \infty}^\infty C_n \cdot e^{i \cdot n \cdot t}$$%ska man ha \cdot mellan int?
Detta går om $f(t)$ är periodisk med perioden $2 \pi$. Om så är fallet gäller:
\begin{align*}
	&f(t) = \sum_{n = - \infty}^\infty C_n \cdot e^{i \cdot n \cdot t} \\
	&f(t) \cdot e^{-imt} = \sum_{n = - \infty}^\infty C_n(e^{i(n - m) \cdot t}) \\
	&\int_{-\pi}^\pi f(t) \cdot e^{-imt}\; dt= \int_{-\pi}^\pi \sum_{n = - \infty}^\infty C_n \cdot e^{i(n - m) \cdot t}\; dt=
	\sum_{n = - \infty}^\infty C_n \cdot \int_{-\pi}^\pi e^{i (n - m) \cdot t}\; dt\\
\end{align*}
Integralen på höger sida $= 0 $ om $n \ne m$ och $= 2\pi $ om $n = m$
$$\int_{-\pi}^\pi f(t) \cdot e^{-i \cdot m \cdot t}\; dt= C_n \cdot 2\pi = C_m \cdot 2\pi \Leftrightarrow C_m = \frac 1 {2\pi} \cdot
\int_{-\pi}^\pi f(t) \cdot e^{i \cdot m \cdot t}\; dt$$
Mer generellt så gäller att om $f(t)$ har en Period $L$ så är:
\begin{align*}
	f(t) &= \sum_{n = - \infty}^\infty C_n \cdot e^{i \cdot n \cdot 2\pi \cdot \frac t L} \\ %att \fraca eller att inte \fraca
	\text{där } C_n &= \frac 1 L \int_{- L / 2}^{L / 2} f(t) \cdot e^{- \frac {i \cdot n \cdot 2\pi} {L}}
\end{align*} %det här måste kollas till. Gränserna på integralen och aligningen på texten samt ekvationerna

Fouriertransformen av en funktion f(t) definieras som:
$$\mathcal{F}(u) = \frac 1 {2\pi} \cdot \int_{-\infty}^\infty f(t) \cdot e^{- i \cdot u \cdot t} dt$$
\pagebreak
\paragraph{Sats 2:} 
$\sum\limits_{n \in \mathbb{Z}} f(n) = \sum\limits_{n \in \mathbb{Z}} \hat{f}(m)$ där $\hat{f}(m)$ är Fouriertransformen till $f(n)$. \\
Vi börjar med att definiera $g(x)$ som $\sum\limits_{n \in \mathbb{Z}} f(x + n)$ eftersom $g(x)$ är periodisk med perioden $1$
kan vi skriva ett uttryck för dess Fourierkoefficienter.
\begin{align*}
	\hat{g_k} &= \int_0^1 \sum_{n \in \mathbb{Z}} f(x + n) \cdot e^{-ikx}\; dx \\
			  &= \sum_{n \in \mathbb{Z}} \int_0^1 f(x + n) \cdot e^{-ikx}\; dx \\
			  &= \sum_{n \in \mathbb{Z}} \int_n^{n + 1} f(x) \cdot e^{-ikx}\; dx \\
			  &= ... \int_{-1}^0 f(x) \cdot e^{-ikx}\; dx + \int_0^1 f(x) - e^{-ikx}\; dx + \int_1^2 f(x) \cdot e^{-ikx}\; dx ... \\
			  &= \int_{- \infty}^\infty f(x) \cdot e^{-ikx}\; dx = \hat{f}(x)
\end{align*}
Som är fouriertransformen av $f(x)$.
$$g(x) = \sum_{n \in \mathbb{Z}} f(x + n) = \sum_{k \in \mathbb{Z}} \hat{f}(k) \cdot e^{ikx}$$%är det inte -ikx?
som är fourierserien till $g(x)$. Om vi nu väljer $x = 0$ får vi:
$$\sum_{n \in \mathbb{Z}} f(n) = \sum_{k \in \mathbb{Z}} \hat{f}(k) \cdot e^0 = \sum_{k \in \mathbb{Z}} \hat{f}(k)$$
$\hfill \qed$
\paragraph{Definition:}
Den kompletta zeta-funktionen.\\
$$\xi (s) = \frac 1 2 \cdot s(s - 1) \cdot \pi^{-s/2} \cdot \Gamma(\frac s 2) \cdot \zeta(s) $$
där
$$\Gamma(s) = \int_0^\infty t^{s - 1} \cdot e^{-t}\; dt$$

\pagebreak
\paragraph{Sats 3:} Den kompletta zeta-funktionens nollställen är nollställen för Riemanns zeta-funktion. \\
\\
Det här är ett väldigt kort bevis. Eftersom varken ${\frac 1 2 \cdot s, (s - 1), \pi^{-s/2}}$ eller $\Gamma(\frac s 2)$
har några nollställen utöver punkterna $s = 0$ och $s = 1$ så måste $\xi(s)$ ha samma nollställen som $\zeta(s)$.

\paragraph{Sats 4:} $\xi(s) = \xi(1 - s)$.\\
$$\Gamma(\frac s 2) = \int_0^\infty t^{s/2 - 1} \cdot e^{-t}\; dt$$
Vi gör substitutionen $t = \pi \cdot n^2 \cdot x$.
$$\Gamma(\frac s 2) = \int_0^\infty \pi^{(s/2 - 1)} \cdot n^{2^{(s / 2 - 1)}} \cdot x^{(s/2 - 1)} \cdot e^{-\pi \cdot
n^2 \cdot x}\; dx$$
$$\pi^{-s/2} \cdot \Gamma(\frac s 2) \cdot n^{-s} = \int_0^\infty x^{s/2 - 1} \cdot e^{-\pi n^2 x}\; dx$$
Vi vet att $\zeta(s) = \sum\limits_{n \ge 1} n^{-s}$
$$\sum_{n \ge 1} \pi^{-s/2} \cdot \Gamma(\frac s 2) \cdot n^{-s} = \pi^{-s/2} \cdot \Gamma(\frac s 2) \cdot \zeta(s) = 
\int_0^\infty x^{s/2 - 1} \cdot \sum_{n \ge 1} e^{-\pi n^2 x}\; dx$$
framöver betecknar vi $\sum\limits_{n \ge 1} e^{-\pi x n^2} \text{ som } \omega(x)$.
Nu delar vi upp integralen vid $1$ och erhåller:%linus skrev får istället för erhåller
$$\int_0^1 x^{s/2 - 1} \cdot \omega(x)\; dx + \int_1^\infty x^{s/2 - 1} \cdot \omega(x)\; dx$$
För att få samma integrationgränser i de båda integralerna gör vi substitutionen $x = x^{-1}$ i den första integralen och erhåller då:
$$\int_1^\infty x^{-s/2 - 1} \cdot \omega(x^{-1})\; dx + \int_1^\infty x^{-s/2 - 1} \cdot \omega(x)\; dx$$
Vi definierar nu funktionen $\theta(x)$ som:
$$\theta(x) = \sum_{n \in \mathbb{Z}} e^{-xn^2\pi}$$
Eftersom $n^2=(-n)^2$ och $e^{-x0^2\pi} = 1$ får vi att 
$$\theta(x) = 2\omega(x) + 1$$%frågan är om man ska ta den inom enkla dollartecken
Vi använder nu Fouriertransformen på $e^{-\pi n^2 x}$ och erhåller
$$\mathcal{F}(e^{-\pi n^2 x})(u) = x^{-1/2} \cdot e^{-\pi u^2/x}$$
Vi använder därefter summationsformeln från Sats 2 och erhåller då:
$$\theta(x) = x^{-1/2}\theta(x^{-1})$$
$$2\omega(x) + 1 = \frac 1 {\sqrt{x}} (2\omega(x^{-1}) + 1) \Leftrightarrow \omega(x^{-1}) = \omega(x) \cdot \sqrt{x} + \frac {\sqrt{x}} 2
- \frac 1 2$$
Om vi nu använder vårt uttryck för $\omega(x^{-1})$ i den tidigare integralen erhåller vi:
$$\int_1^\infty x^{-s/2 - 1} \cdot \omega(x^{-1})\; dx = - \frac 1 s + \frac 1 {s - 1} + \int_1^\infty x^{(s + 1)/2} \cdot \omega(x)\; dx$$
Och hela ekvationen blir då:
$$\xi(s) = \pi^{-s/2} \cdot \Gamma(\frac s 2) \cdot \zeta(s) = - \frac 1 {s(1 - s)} + \int_1^\infty(x^{s/2 - 1} + x^{-(s + 1)/2}) \cdot
\omega(x)\; dx$$
Om vi byter ut $s$ mot $(1 - s)$ i det högra uttrycket erhåller vi samma uttryck igen. Därför gäller det att $\xi(s) = \xi(1 - s)$ \\
\hfill \qed

\paragraph{Sats 5:} De triviala nollställena till Riemanns zeta-funktion är de enda nollställena utanför intervallet $0 < Re(s) < 1$. \\
\\ %Skrivas i obestämd form ex. I Sats 1 bevisades att ...?
I Sats 1 bevisade jag att $\zeta(s)$ inte har några nollställen för $\zeta(s)$ och i Sats 4 bevisade jag att $\xi(s) = \xi(1 - s)$.
Om vi antar att det finns en punkt p med $Re(p) < 0$ sådan att $\zeta(p) = 0$ innebär det av Sats 3 att $\xi(p) = 0 \Leftrightarrow
\xi(1 - p) = 0 \Leftrightarrow \zeta(1 - p) = 0$ men om $Re(p) < 0$ är $Re(1 - p) > 1$. Då innebär det att det finns en punkt
$q = 1 - p$ med $Re(q) > 1$ sådan att $\zeta(q) = 0$. Men detta strider mot Sats 1. Det kan alltså inte finnas en punkt $p$ med
$Re(p) < 0$ sådan att $\zeta(p) = 0$. \\
\hfill \qed %Ska skrivas om så att ekvationerna inte linebreakas
%I slutet på andra meningen i stas 5 så ska det stå "med Re > 1" efter zetat. vilket zeta?

\paragraph{Sats 6:} $Re(ln(-a)) = ln(a)$ \\
Vi börjar med Eulers identitet:
\begin{align*}
	e^{i\pi} &= -1 \Leftrightarrow -a = a \cdot e^{i\pi} \\
			 &= e^{ln(a)} \cdot e^{i\pi} \\
			 &= e^{ln(a) + i\pi} \\
			 &= e^{ln(-a)} \Leftrightarrow ln(-a) \\
			 &= ln(a) + i\pi 
\end{align*}
$$Re(ln(a) + i\pi) = ln(a) = Re(ln(-a))$$
\hfill \qed

\pagebreak

\paragraph{Sats 7:} $ln(1 - i) = - \sum\limits_{n = 1}^\infty \frac {x^n} n$
$$f(x) = ln(1 - x) \Rightarrow f(0) = ln(1) = 0$$
Definitionen av Maclaurinpolynom är:
$$f(x) = \sum_{n = 0}^\infty \frac {f^{(n)}(0) \cdot x^n} {n!}$$
men då $ln(1) = 0$ är:
$$ln(1 - x) = f(x) = \sum_{n = 1}^{\infty} \frac {f^{(n)}(0) \cdot x^n} {n!}$$
Derivering ger att
\begin{align*}
	&f^{(n)}(0) = -(n - 1)! \\
	&\Downarrow \\ 
	&ln(1 - x) = \sum_{n = 1}^\infty \frac {-(n - 1)! \cdot x^n} {n!} = - \sum_{n = 1}^\infty \frac {x^n} n
\end{align*}
\hfill \qed

\paragraph{Sats 8:} $\zeta(s + it) = 0$ for $s = 1$ omm $\zeta(s)^3 \cdot \zeta(s + it)^4 \cdot \zeta(s + 2it) = 0$
för $s = 1$. \\
\\
$\zeta(s)$ har en singularitet för $s = 1$. Men eftersom faktorn $\zeta(s + it)^4$ är av högre grad kan singulariteterna
för $\zeta(s)$ inte ta ut ett eventuellt nollställe. Termen $\zeta(s + 2it)$ kan bli $0$, men då den aldrig kan bli en 
singularitet kan den inte ta ut ett eventuellt nollställe.\\

\newcommand*{\ps}{1 - p^{-s}}
\paragraph{Sats 9:} $\zeta(s)$ har inga nollställen för $Re(s) = 1$. \\
\\
Vi använder oss av ett uttryck från Sats 1:
$$\zeta(s) = \prod_p \frac 1 {1 - p^{-s}}$$
$$ln(\frac 1 {1 - p^{-s}}) = ln(1) - ln(1 - p^{-s}) = -ln(1 - p^{-s})$$
\[ ln \left | \zeta(s) \right| = ln \left |\prod_p \frac 1 {1 - p^{-s}} \right |
	= \sum_p ln \left | \frac 1 {1 - p^{-s}} \right | = - \sum_p ln \left |1 - p^{-s} \right | \]
Sats 6 ger nu:
\[-\sum_p ln \left | \ps \right | = - Re(\sum_p ln(\ps)) \]
Sats 7 ger nu:
$$-Re(\sum_p ln(\ps)) = Re(\sum_p \sum_{n = 1}^\infty(\frac {p^{-ns}} n))$$
Vi definierar nu $h(x) = \zeta(x)^3 \cdot \zeta(x + it)^4 \cdot \zeta(x + 2it)$ logaritmlagarna ger nu:
\begin{align*}
	ln(h(x)) &= 3 ln \left | \zeta(x) \right | + 4 ln \left | \zeta(x + it) \right | + ln \left | \zeta(x + 2it) \right |\\  
			 &= \sum_p \sum_{n = 1}^\infty \frac 1 n \cdot p^{nx} \cdot Re(3 + 4p^{-int} + p^{-int \cdot 2}) 
\end{align*}
$$\frac 1 n \cdot p^{-nx} \ge 0$$
vilket innebär att det vi måste undrsöka är om \\
$$Re(3 + 4p^{-int} + p^{-2int}) \ge 0$$.
\\
$$p^{-int} = e^{ln(p) \cdot -int}$$
$$\theta = -ln(p) \cdot n \cdot t \Rightarrow e^{i\theta} \Rightarrow Re(3 + 4 \cdot e^{i\theta} + e^{i2\theta})$$
Eulers formel ger nu:
$$Re(3 + 4e^{i\theta} + e^{2i\theta}) = 3 + 4cos(\theta) + cos(2\theta)$$
likheten $cos(2\theta) = 2cos^2(\theta) - 1$ ger:
$$3 + 4 cos(\theta) + 2 cos^2(\theta) - 1 = 2(1 + 2cos(\theta) + cos^2(\theta)) = 2(1 + cos(\theta))^2 \ge 0$$
alla termer i vår summa är alltså större eller lika med 0. Då exempelvis\\
$\theta = -ln(2) \cdot 2$ ger en term $ > 0$ innebär detta att summan är strikt större än 0.\\
%w00t? nej om $f(x) = \sum_0^\infty g(x), g(x) \ge 0 är f(x) också \ge 0$
Det finns alltså inga nollställen för $\zeta(s)$ med $Re(s) = 1$.\\
\hfill \qed

\paragraph{Sats 10:} Det finns inga nollställen för $\zeta(s)$ med $Re(s) = 0.$\\
\\
Sats 3 och 4 visar att om $\zeta(s)$ är ett nollställe är $\zeta(1 - s)$ också det.
Om det finns något $\zeta(0 + it) = 0$ följer även att $\zeta(1 - (0 + it)) = \zeta(1 - it) = 0$. Men detta 
motbevisades i Sats 9.\\
Notera att Sats 4 inte fungerar i punkterna $s = 1$ och $s = 0$. Med då $\zeta(0) = -\frac 1 2$ och 
$\zeta(1) = \infty$ håller satsen.
\hfill \qed

\paragraph{Sats 11:} Om $\zeta(s) = 0$ för något tal $s = \frac 1 2 + it$ med $t \in \mathbb{R}$ är $\zeta(\overline{s}) = 0$ \\
\\
Det har tidigare visats att om $\zeta(s) = 0$ är $\zeta(1 - s) = 0$. $\zeta(\frac 1 2 + it) = 0 \Leftrightarrow 
\zeta(1 - (\frac 1 2 + it)) = \zeta(\frac 1 2 - it) = 0$
\hfill \qed

\paragraph{Definition: Mellintransform} $\{\mathcal{M}f\}(s) = \varphi(s) = \int_0^\infty x^{s - 1} \cdot f(x) \; dx $ \\
\\
Mellintransformen har inversionsformen
\[
	f(x) = \frac {1} {2 \pi i} \cdot \int_{C - i \infty}^{C + i \infty} x^{-s} \cdot \varphi(s) \; ds
\]

\paragraph{Sats 12:} 
\[
	\{\mathcal{M} \omega \} (x) = \zeta(2s) \cdot \Gamma(s) \cdot \pi^{-s}
\]
med 
\[
	\omega(y) = \sum\limits_{n \leq 1} e^{-n^2\pi y}
\]
\\ 
Denna likhet härleddes i Sats 4, men då utan beteckningen Mellintransform.\\
\\
\paragraph{Sats 13:} 
\[
	\omega(y) = \frac {1} {2\pi i} \int_{C - i\infty}^{C - i \infty} \zeta(zs) \cdot \Gamma(s) \cdot \pi^{-s} \cdot
		y^{-s} \; ds \text{ för } c > \frac 1 2
\] 
Detta följer av att applicera Mellintransformens inversionsformel i Sats 12.\\

\paragraph{Definition:} $\Xi(t) = \zeta(\frac 1 2 + it)$\\
\\
\paragraph{Sats 14:} 
\[
	\int_0^\infty (t^2 + \frac 1 4)^{-1} \cdot \Xi(t) \cdot \cos(xt) \; dt = \frac 1 2 \pi(e^{\tfrac 1 2 x} -
		2e^{-\tfrac 1 2 x} \omega(e^{- 2 x}))
\]
\[
	\text{Sätt } Q(x) = \int_0^\infty (t^2 + \frac 1 4)^{-1} \cdot \Xi(t) \cdot \cos(xt) \; dt
\]
Då $t^2 + \frac 1 4) \cdot \Xi(t) \cdot \cos(xt)$ är symmetrisk kring $t = 0$ (se Sats 11 för $\Xi(t)$'s symmetri)
så gäller:
\[
	Q(x) = \frac 1 2 \int_{-\infty}^\infty (t^2 + \frac 1 4)^{- 1} \cdot \Xi(t) \cdot \cos(xt) \; dt \quad (1)
\]
Då $\sin(xt)$ är ``inverst symmetrisk'' kring $t = 0$ följer det att 
\[
	\frac 1 2 \int_{-\infty}^\infty (t^2 + \frac 1 4)^{- 1} \cdot \Xi(t) \cdot i\sin(xt) \; dt = 0 \quad (2)
\]
Attidion av $(1)$ och (2) tillsammans med Eulers formel ger nu: 
\[
	Q(x) = \frac 1 2 \cdot \int_{- \infty}^\infty (t^2 + \frac 1 4)^{-1} \Xi(t)\cdot e^{ixt} \; dt
\]
låt $s = \frac 1 2 + it$. Då följer:
\[
	Q(x) = \frac 
	{
		e^{- \tfrac 1 2 x t}
	}
	{
		2i
	}
	\cdot \int_{\tfrac 1 2 - \infty}^{\tfrac 1 2 + \infty} \frac {1} {s(1-s)} \Xi(s) \cdot e^{xs} \; ds
\]
Definitionen av $\zeta(s)$ ger nu:
\[
	Q(x) = - \frac {e^{- \frac 1 2 xt}} {4i} \cdot \int_{\tfrac 1 2 - \infty}^{\tfrac 1 2 + \infty} \zeta(s) \cdot \Gamma
		\pi^{- \frac s 2} \cdot e^{xs} \; ds
\]
$z(s) \cdot \Gamma(\frac s 2) \pi^{- \frac s 0} \cdot e^{xs}$ har singulariteter i $s = 0$ och $s = 1$.
Därmed följer det att om vi flyttar linjeintegralen till höger om linjen $c=1$ så korrigeras förändringen med att 
subtrahera residualen i $s = 1$ så för $c > 1$ gäller:
\[
	Q(x) = - \frac {e^{- \frac 1 2 xt}} {4i} \cdot \int_{\tfrac 1 2 - \infty}^{\tfrac 1 2 + \infty} \zeta(s) \cdot \Gamma
		\pi^{- \frac s 2} \cdot e^{xs} \; ds + \frac {e^{- \frac 1 2 xt}} {4i} \cdot2 \pi i \operatorname{Res}(\zeta(s) \cdot
		\Gamma(\frac s 2) \cdot \pi^{-\frac s 2} \cdot e^{xs}, 1)
\]
Då $\operatorname{Res}(\zeta(s), 1) = 1$ (se Appendix Sats 15) följer det att:
\[
	\frac {e^{- \frac 1 2 x}} {4i} \cdot 2 \pi i \cdot 1 \cdot \Gamma(\frac 1 2) \cdot \pi^{- \frac 1 2} \cdot e^x =
		\frac {e^{\frac 1 2 x}} {2} \cdot \pi 1 \cdot \sqrt{x} \cdot \frac {1} {\sqrt{x}} = \frac {e^{\frac {1 2} x} \pi} {2}
\]
Därmed följer:
\[
	Q(x) = - \frac {e^{-\frac 1 2 x}} {4 i} \cdot \int_{C - \infty}^{C + \infty} \zeta(s) \cdot \Gamma(\frac s 2)
		\cdot \pi^{- \frac 1 2} \cdot e^{xs} \; ds + \frac \pi 2 \cdot e^{\frac 1 2 x}
\]
genom att sätta $x = e^{-2x}$ fås det från Sats 13 att:
\[
	Q(x) = - \pi e^{- \frac 1 2 x} \cdot \omega(e^{- 2 x}) + \frac \pi 2 e^{\frac 1 2 x} = \frac 1 2 \pi
		(e^{\frac 1 2 x} - 2 e^{- \frac 1 2 x} \cdot \omega(e^{-2x}))
\]
\hfill \qed

\paragraph{Sats 15:} För alla heltal $n$ gäller:
\[%vad är det här?
	\lim_{a \to \frac n 4 \pi^+} \frac {
		d^{2n} [
			e^{\frac 1 2 ia} (\frac 1 2 + \omega(e^{2ia}))
		]}
		{
			da^{2n}
		} = 0
\]
Observera att:
\begin{align*}
	\omega(i + v) &= \sum_{n = 1}^\infty e^{-n^2 \pi(i + v)} \\
		&= \sum_{n = 1}^\infty e^{-n^2 v \pi} \cdot e^{-\pi i n^2} \\
		&= \sum_{n = 1}^\infty e^{-n^2 v \pi} \cdot (- 1)^{n^2} \\
		&= \sum_{n = 1}^\infty e^{-n^2 v \pi} \cdot (- 1)^n
\end{align*}
Det sista steget gäller då udda $n$ ger udda $n^2$ och jämna $n$ ger jämna $n^2$.
\[
	2 \omega(4v) - \omega(v) = \sum_{n = 1}^\infty 2 e^{-n^2 \pi 4 v} - e^{-n^2 \pi v} = \sum_{n = 1}^\infty 2
		e^{-(2n)^2\pi v} - e^{-n^2 \pi v} 
\]
\[
	= \sum_{n = 1}^\infty (-1)^n \cdot e^{-n^2 vt} = \omega(i + v) = 2 \omega(4 v) -
			\omega(v) \qquad (1)
\]
\[
	\omega(x) x^{- \frac 1 2} \cdot \omega(\frac 1 x) + \frac 1 2  x^{- \frac 1 2} - \frac 1 2 \text{ Se Sats 4 för härledning}
\]
Vilket ger att $(1)$ blir:
\[
	\omega(i + v) = \frac {1} {\sqrt v} \cdot \omega(\frac {1} {4v}) - \frac {1} {\sqrt v} \cdot \omega(\frac 1 v) - \frac 1 2
		\qquad (2)
\]
Om man nu skriver ut summorna ser man att $\omega(i + v) + \frac 1 2$ och alla dess derivator går
vilket även innebär att de går mod $0$ längs all vinklar $|\operatorname{arg}(z)| < \frac 1 2 \pi$ då vi för något $v$ med 
$\operatorname{Re}(v) > 0$ har att:
\[
	\left |
		 \sum_{n = 1}^\infty e^{-\pi n^2 \frac 1 v} 
	\right | \leq 
		\sum_{n = 1}^\infty e^{-\pi n^2 \frac {\operatorname{Re}(v)} {|v|}}
	\leq
		\sum_{n = 1}^\infty e^{-\pi n^2 \frac {1} {|v|}}.
\]
Då $a \to \frac {\pi^+} {4}$ antyder att $e^{2ia} \to i$ längs alla ``vägar''
med $|\operatorname{arg}(e^{2\pi i} - i) | < \frac 1 2 \pi$, så bevisar detta satsen\\
\hfill \qed
\paragraph{Sats 16:} Riemanns Zeta-funktion har oändligt många nollställen längst linjen $\frac 1 2 + it$
\\
Genom att substituera $x = -ia$ i Sats 14
så fås uttrycket:
\[
	\int_0^\infty (t^2 + \frac 1 4)^{-1} \Xi(t) \cdot \cosh(at) \; dt = \frac \pi 2 (e^{- \frac 1 2 ia} - 2 e^{\frac 1 2 ia}
		\cdot \omega(e^{2ia}))
\]
\[
	= \pi \cos(\frac a 2) - \pi e^{\frac 1 2 ia} (\frac 1 2 + \omega(e^{2 ia})) \qquad (1)
\]
\[
	\text{där } \cosh(at) = \frac 1 2 (e^{-at} + e^{at})
\]
om man nu deriverar $(1)$ $2n$ antal gånger med avseende på a får man:
\[
	\int_0^\infty (t^2 + \frac 1 4)^{-1} \cdot t^{2n} \cdot \Xi(t) \cdot \cosh(at) \; dt = \frac {\pi(-1)^n}{2^{2n}}
		\cos(\frac a 2) - \frac {\pi d^{2n}} {da^{2n}} \left [
			e^{\frac 1 2 ia} (\frac 1 2 + \omega(e^{2ia}))
		\right ]
\]
$a$ får nu gå mot $\frac 1 4 \pi^+$.
Sats 15 innebär nu att den sista tremen i det högra ledet $\to 0$, vilket ger:
\[
	\lim_{a \to \frac 1 4 \pi^+} \int_0^\infty (t^2 + \frac 1 4)^{-1} \cdot t^{2n}\Xi(t) \cdot \cosh(at) \; dt = 
		\frac {\pi (-1)^n} {2^{2n}} \cos(\frac 8 \pi). \qquad (2)
\]
Detta innebär att det vänstra ledet växlar tecken oändligt många gånger. Då vänstra ledet är kontinuerligt
längs linjen $\frac 1 2 + it$ innebär detta att det även blir lika med $0$ oändligt många gger.
Och dåingen av de andra faktorerna blir $0$ oändligt många gånger betyder detta att $\Xi(t)$ har oändligt många nollställen.\\
\hfill \qed
\end{document}
