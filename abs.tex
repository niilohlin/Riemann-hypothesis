\documentclass{article}%[fleqn] kan vara snyggt
\usepackage{amsthm}
\usepackage{amsmath}
\usepackage{amsfonts}
\usepackage{graphicx}
%\usepackage{mathtools}
\usepackage[swedish]{babel}
\usepackage[utf8]{inputenc}

\def \width {3mm}

\DeclarePairedDelimiter\abs{\lvert}{\rvert}
\DeclarePairedDelimiter\norm{\lVert}{\rVert}

%Swap the definiteon of \abs* and \norm*, so that \abs
%and \nomr resizes the size of the brackets, and the
%starred version does not.
\makeatletter
\let\oldabs\abs
\def\abs{\@ifstar{\oldabs}{\oldabs*}}
\makeanother

\begin{document}
\title{p100}
\author{Linus Bergkvist}
\date{}
\maketitle
%\setlength\parindent{0pt}
%\paragraph{Introduktion:} 
\noindent
{\bf Introduktion}\\
	Riemannhypotesen går ut på att alla icke-triviala nollställen till Riemanns
	zeta-funktionligger på linjen $\frac 1 2 + i \cdot t$. De triviala
	nollställena är $-2, -4, -6, -8, -10 \ldots$

\paragraph{Sats 1:}
Det finns inga nollställen till Riemanns zeta-funktion med $Re > 1$. \\
\\
För $s$ med $Re(s) > 1$ definieras Riemanns zeta-funktion som 
$\zeta(s) = \sum_{n = 1}^\infty \frac {1} {n^s}$ %ska man ta och sätta \limits_? för snygghetens skull?
\begin{equation}
	\begin{aligned}
		&\zeta(s) = \frac 1 {1^s} + \frac 1 {2^s} + \frac 1 {3^s} + \frac 1 {4^s} ... \notag\\
		&\zeta(s) \cdot \frac 1 {2^s} = \frac 1 {2^s} + \frac 1 {4^s} + \frac 1 {6^s} + \frac 1 {8^s} ... \\
		&\zeta(s) - \frac {\zeta(s)} {2^s} = \frac 1 {1^s} + \frac 1 {3^s} + \frac 1 {5^s} ... &
		\hspace{\width} &\text{Vi har nu tagit bort alla multiplar av } \frac 1 {2^s} \\
		&\zeta(s) (1 - \frac 1 {2^s}) \cdot \frac 1 {3^s} = \frac 1 {3^s} + \frac 1 {9^s} + \frac 1 {15^s} ...\\
		&\zeta(s)(1 - \frac 1 {2^s})(1 - \frac 1 {3^s}) = \frac 1 {1^s} + \frac 1 {5^s} + \frac 1 {7^s} ... &
		\hspace{\width} &\text{Vi har nu tagit bort alla} \\
		& & \hspace{\width} &\text{återstående multiplar av } \frac 1 {3^s} \\
	\end{aligned}
	\phantom{\hspace{4cm}}
\end{equation}
Om man gör så med alla primtal får man: 
\begin{equation}
	\begin{aligned}
		\notag
		&\zeta(s) \cdot (1 - \frac 1 {2^s})(1 - \frac 1 {3^s})(1 - \frac 1 {5^s}) ... = \frac 1 {1^s} = 1 \Leftrightarrow \\  
		&\zeta(s)  = \frac 1 {(1 - \frac 1 {2^s})(1 - \frac 1 {3^s})...} = \prod_p \frac 1 {1 - p^{-s}}.\hspace{\width} Re(s) > 1 \\
		&\frac {\zeta(s)} {\prod_p \frac 1 {(1 - p^{-s})}} = 1 = \zeta(s) \cdot \prod_p (1 - p^{-s}) \\
	\end{aligned}
\end{equation}
\\
Vilket visar att $\zeta(s) \ne 0 \text{ För } Re(s) > 1$

\pagebreak
\paragraph{Definition: Fourierserie och Fouriertransform.\\}
Eulers formel ger att $e^{i \cdot x} = cos(x) + i \cdot sin(x)$ Men går det att uttrycka en funktion som en oändlig summa
med sinus och cosinusvågor?\\
Mer matematisk blir frågan om kan man skriva en funktion $f(t)$ som 
$$f(t) = \sum_{n = - \infty}^\infty C_n \cdot e^{i \cdot n \cdot t}$$%ska man ha \cdot mellan int?
Detta går om $f(t)$ är periodisk med perioden $2 \pi$. Om så är fallet gäller:
\begin{align*}
	&f(t) = \sum_{n = - \infty}^\infty C_n \cdot e^{i \cdot n \cdot t} \\
	&f(t) \cdot e^{-imt} = \sum_{n = - \infty}^\infty C_n(e^{i(n - m) \cdot t}) \\
	&\int_{-\pi}^\pi f(t) \cdot e^{-imt}\; dt= \int_{-\pi}^\pi \sum_{n = - \infty}^\infty C_n \cdot e^{i(n - m) \cdot t}\; dt=
	\sum_{n = - \infty}^\infty C_n \cdot \int_{-\pi}^\pi e^{i (n - m) \cdot t}\; dt\\
\end{align*}
Integralen på höger sida $= 0 $ om $n \ne m$ och $= 2\pi $ om $n = m$
$$\int_{-\pi}^\pi f(t) \cdot e^{-i \cdot m \cdot t}\; dt= C_n \cdot 2\pi = C_m \cdot 2\pi \Leftrightarrow C_m = \frac 1 {2\pi} \cdot
\int_{-\pi}^\pi f(t) \cdot e^{i \cdot m \cdot t}\; dt$$
Mer generellt så gäller att om $f(t)$ har en Period $L$ så är:
\begin{align*}
	f(t) &= \sum_{n = - \infty}^\infty C_n \cdot e^{i \cdot n \cdot 2\pi \cdot \frac t L} \\ %att \fraca eller att inte \fraca
	\text{där } C_n &= \frac 1 L \int_{- L / 2}^{L / 2} f(t) \cdot e^{- \frac {i \cdot n \cdot 2\pi} {L}}
\end{align*} %det här måste kollas till. Gränserna på integralen och aligningen på texten samt ekvationerna

Fouriertransformen av en funktion f(t) definieras som:
$$\mathcal{F}(u) = \frac 1 {2\pi} \cdot \int_{-\infty}^\infty f(t) \cdot e^{- i \cdot u \cdot t} dt$$
\pagebreak
\paragraph{Sats 2:} 
$\sum\limits_{n \in \mathbb{Z}} f(n) = \sum\limits_{n \in \mathbb{Z}} \hat{f}(m)$ där $\hat{f}(m)$ är Fouriertransformen till $f(n)$. \\
Vi börjar med att definiera $g(x)$ som $\sum\limits_{n \in \mathbb{Z}} f(x + n)$ eftersom $g(x)$ är periodisk med perioden $1$
kan vi skriva ett uttryck för dess Fourierkoefficienter.
\begin{align*}
	\hat{g_k} &= \int_0^1 \sum_{n \in \mathbb{Z}} f(x + n) \cdot e^{-ikx}\; dx \\
			  &= \sum_{n \in \mathbb{Z}} \int_0^1 f(x + n) \cdot e^{-ikx}\; dx \\
			  &= \sum_{n \in \mathbb{Z}} \int_n^{n + 1} f(x) \cdot e^{-ikx}\; dx \\
			  &= ... \int_{-1}^0 f(x) \cdot e^{-ikx}\; dx + \int_0^1 f(x) - e^{-ikx}\; dx + \int_1^2 f(x) \cdot e^{-ikx}\; dx ... \\
			  &= \int_{- \infty}^\infty f(x) \cdot e^{-ikx}\; dx = \hat{f}(x)
\end{align*}
Som är fouriertransformen av $f(x)$.
$$g(x) = \sum_{n \in \mathbb{Z}} f(x + n) = \sum_{k \in \mathbb{Z}} \hat{f}(k) \cdot e^{ikx}$$%är det inte -ikx?
som är fourierserien till $g(x)$. Om vi nu väljer $x = 0$ får vi:
$$\sum_{n \in \mathbb{Z}} f(n) = \sum_{k \in \mathbb{Z}} \hat{f}(k) \cdot e^0 = \sum_{k \in \mathbb{Z}} \hat{f}(k)$$
$\hfill \qed$
\paragraph{Definition:}
Den kompletta zeta-funktionen.\\
$$\xi (s) = \frac 1 2 \cdot s(s - 1) \cdot \pi^{-s/2} \cdot \Gamma(\frac s 2) \cdot \zeta(s)$$
där
$$\Gamma(s) = \int_0^\infty t^{s - 1} \cdot e^{-t}\; dt$$

\pagebreak
\paragraph{Sats 3:} Den kompletta zeta-funktionens nollställen är nollställen för Riemanns zeta-funktion. \\
\\
Det här är ett väldigt kort bevis. Eftersom varken ${\frac 1 2 \cdot s, (s - 1), \pi^{-s/2}}$ eller $\Gamma(\frac s 2)$
har några nollställen utöver punkterna $s = 0$ och $s = 1$ så måste $\xi(s)$ ha samma nollställen som $\zeta(s)$.

\paragraph{Sats 4:} $\xi(s) = \xi(1 - s)$.\\
$$\Gamma(\frac s 2) = \int_0^\infty t^{s/2 - 1} \cdot e^{-t}\; dt$$
Vi gör substitutionen $t = \pi \cdot n^2 \cdot x$.
$$\Gamma(\frac s 2) = \int_0^\infty \pi^{(s/2 - 1)} \cdot n^{2^{(s / 2 - 1)}} \cdot x^{(s/2 - 1)} \cdot e^{-\pi \cdot
n^2 \cdot x}\; dx$$
$$\pi^{-s/2} \cdot \Gamma(\frac s 2) \cdot n^{-s} = \int_0^\infty x^{s/2 - 1} \cdot e^{-\pi n^2 x}\; dx$$
Vi vet att $\zeta(s) = \sum\limits_{n \ge 1} n^{-s}$
$$\sum_{n \ge 1} \pi^{-s/2} \cdot \Gamma(\frac s 2) \cdot n^{-s} = \pi^{-s/2} \cdot \Gamma(\frac s 2) \cdot \zeta(s) = 
\int_0^\infty x^{s/2 - 1} \cdot \sum_{n \ge 1} e^{-\pi n^2 x}\; dx$$
framöver betecknar vi $\sum\limits_{n \ge 1} e^{-\pi x n^2} \text{ som } \omega(x)$.
Nu delar vi upp integralen vid $1$ och erhåller:%linus skrev får istället för erhåller
$$\int_0^1 x^{s/2 - 1} \cdot \omega(x)\; dx + \int_1^\infty x^{s/2 - 1} \cdot \omega(x)\; dx$$
För att få samma integrationgränser i de båda integralerna gör vi substitutionen $x = x^{-1}$ i den första integralen och erhåller då:
$$\int_1^\infty x^{-s/2 - 1} \cdot \omega(x^{-1})\; dx + \int_1^\infty x^{-s/2 - 1} \cdot \omega(x)\; dx$$
Vi definierar nu funktionen $\theta(x)$ som:
$$\theta(x) = \sum_{n \in \mathbb{Z}} e^{-xn^2\pi}$$
Eftersom $n^2=(-n)^2$ och $e^{-x0^2\pi} = 1$ får vi att 
$$\theta(x) = 2\omega(x) + 1$$%frågan är om man ska ta den inom enkla dollartecken
Vi använder nu Fouriertransformen på $e^{-\pi n^2 x}$ och erhåller
$$\mathcal{F}(e^{-\pi n^2 x})(u) = x^{-1/2} \cdot e^{-\pi u^2/x}$$
Vi använder därefter summationsformeln från Sats 2 och erhåller då:
$$\theta(x) = x^{-1/2}\theta(x^{-1})$$
$$2\omega(x) + 1 = \frac 1 {\sqrt{x}} (2\omega(x^{-1}) + 1) \Leftrightarrow \omega(x^{-1}) = \omega(x) \cdot \sqrt{x} + \frac {\sqrt{x}} 2
- \frac 1 2$$
Om vi nu använder vårt uttryck för $\omega(x^{-1})$ i den tidigare integralen erhåller vi:
$$\int_1^\infty x^{-s/2 - 1} \cdot \omega(x^{-1})\; dx = - \frac 1 s + \frac 1 {s - 1} + \int_1^\infty x^{(s + 1)/2} \cdot \omega(x)\; dx$$
Och hela ekvationen blir då:
$$\xi(s) = \pi^{-s/2} \cdot \Gamma(\frac s 2) \cdot \zeta(s) = - \frac 1 {s(1 - s)} + \int_1^\infty(x^{s/2 - 1} + x^{-(s + 1)/2}) \cdot
\omega(x)\; dx$$
Om vi byter ut $s$ mot $(1 - s)$ i det högra uttrycket erhåller vi samma uttryck igen. Därför gäller det att $\xi(s) = \xi(1 - s)$ \\
\hfill \qed

\paragraph{Sats 5:} De triviala nollställena till Riemanns zeta-funktion är de enda nollställena utanför intervallet $0 < Re(s) < 1$. \\
\\ %Skrivas i obestämd form ex. I Sats 1 bevisades att ...?
I Sats 1 bevisade jag att $\zeta(s)$ inte har några nollställen för $\zeta(s)$ och i Sats 4 bevisade jag att $\xi(s) = \xi(1 - s)$.
Om vi antar att det finns en punkt p med $Re(p) < 0$ sådan att $\zeta(p) = 0$ innebär det av Sats 3 att $\xi(p) = 0 \Leftrightarrow
\xi(1 - p) = 0 \Leftrightarrow \zeta(1 - p) = 0$ men om $Re(p) < 0$ är $Re(1 - p) > 1$. Då innebär det att det finns en punkt
$q = 1 - p$ med $Re(q) > 1$ sådan att $\zeta(q) = 0$. Men detta strider mot Sats 1. Det kan alltså inte finnas en punkt $p$ med
$Re(p) < 0$ sådan att $\zeta(p) = 0$. \\
\hfill \qed %Ska skrivas om så att ekvationerna inte linebreakas
%I slutet på andra meningen i stas 5 så ska det stå "med Re > 1" efter zetat. vilket zeta?

\paragraph{Sats 6:} $Re(ln(-a)) = ln(a)$ \\
Vi börjar med Eulers identitet:
\begin{align*}
	e^{i\pi} &= -1 \Leftrightarrow -a = a \cdot e^{i\pi} \\
			 &= e^{ln(a)} \cdot e^{i\pi} \\
			 &= e^{ln(a) + i\pi} \\
			 &= e^{ln(-a)} \Leftrightarrow ln(-a) \\
			 &= ln(a) + i\pi 
\end{align*}
$$Re(ln(a) + i\pi) = ln(a) = Re(ln(-a))$$
\hfill \qed

\pagebreak

\paragraph{Sats 7:} $ln(1 - i) = - \sum\limits_{n = 1}^\infty \frac {x^n} n$
$$f(x) = ln(1 - x) \Rightarrow f(0) = ln(1) = 0$$
Definitionen av Maclaurinpolynom är:
$$f(x) = \sum_{n = 0}^\infty \frac {f^{(n)}(0) \cdot x^n} {n!}$$
men då $ln(1) = 0$ är:
$$ln(1 - x) = f(x) = \sum_{n = 1}^{\infty} \frac {f^{(n)}(0) \cdot x^n} {n!}$$
Derivering ger att
\begin{align*}
	&f^{(n)}(0) = -(n - 1)! \\
	&\Downarrow \\ 
	&ln(1 - x) = \sum_{n = 1}^\infty \frac {-(n - 1)! \cdot x^n} {n!} = - \sum_{n = 1}^\infty \frac {x^n} n
\end{align*}
\hfill \qed

\paragraph{Sats 8:} $\zeta(s + it) = 0$ for $s = 1$ omm $\zeta(s)^3 \cdot \zeta(s + it)^4 \cdot \zeta(s + 2it) = 0$
för $s = 1$. \\
\\
$\zeta(s)$ har en singularitet för $s = 1$. Men eftersom faktorn $\zeta(s + it)^4$ är av högre grad kan singulariteterna
för $\zeta(s)$ inte ta ut ett eventuellt nollställe. Termen $\zeta(s + 2it)$ kan bli $0$, men då den aldrig kan bli en 
singularitet kan den inte ta ut ett eventuellt nollställe.\\

\paragraph{Sats 9:} $\zeta(s)$ har inga nollställen för $Re(s) = 1$. \\
\\
Vi använder oss av ett uttryck från Sats 1:
$$\zeta(s) = \prod_p \frac 1 {1 - p^{-s}}$$
$$ln(\frac 1 {1 - p^{-s}}) = ln(1) - ln(1 - p^{-s}) = -ln(1 - p^{-s})$$
$$ln\abs{\zeta(s)}$$
\end{document}
